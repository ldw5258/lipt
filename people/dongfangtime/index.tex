\documentclass[a4paper,11pt]{article}
\input{add.to}
\tolerance=1000
\providecommand{\alert}[1]{\textbf{#1}}

\title{【东方时代环球时事解读·时事节简版】}
\author{东方评论员}
\date{02 August 2011}

\begin{document}

\maketitle

\setcounter{tocdepth}{1}
\tableofcontents
\vspace*{1cm}

\section{2005年02月28日 星期一}
\label{sec-1}
\begin{itemize}

\item 日本媒体称:朝鲜拟6月重返六方会谈 10月与美签署条约\\
\label{sec-1_1}%
【东京消息】据日本产经新闻报道,朝鲜将会在6月重返朝核六方会谈的谈判桌,并会在10月与美国签署一项条约。消息是透过非官方渠道传达给韩国政府,然后再传达给日本政府。不过消息还未得到日本官员的证实。

  据日本产经新闻28日引用日本政府有关人士的话报道,朝鲜通过非正式渠道向韩国政府转达了这一信息。据推测,由于重启六方会谈之前需要进行工作组会议,朝鲜很有可能3月份正式表明重返六方会谈的意向。

  据日本政府有关人士透露,朝鲜向韩国政府转达了第四次六方会谈中谈论朝鲜半岛实现无核化以及确定经济支援日程后,10月份为止与美国签定具体协定的意向。

  【时事点评】我们知道,早在2000年,北京在公布《一个中国原则与台湾问题》白皮书中,就已非常明确指出三种可能导致大陆采用断然措施的情况,包括使用武力。白皮书明确指出:“如果出现台湾被以任何名义从中国分割出去的重大事变,如果出现外国侵占台湾,如果台湾当局无限期地拒绝通过谈判和平解决两岸统一问题,中国政府只能被迫采取一切可能的断然措施、包括使用武力,来维护中国的主权和领土完整,完成中国的统一大业。”外界随后将此视为大陆的 “动武三前提”。

  在东方评论员看来,也就是这么个“动武三前提”,尽管全是原则性的东西,操作性不佳,但是,不论是“台独”、还是支持台独的势力就是一直把它“当回事”。

\begin{itemize}

\item 美国国内有一批人也一直在谋求着做这么一件事\\
\label{sec-1_1_1}%
事实上,我们认为,长久以来,围绕朝核问题,美国国内有一批人也一直在谋求着做这么一件事,即,为朝核问题“划定一条坚守的红线”、其性质就如中国在台独问题上清楚地划出的“那条不可逾越的红线一样”。

    在我们看来,美国的新保守主义者不仅一直计划这样做,并前后相继做了多次努力,尝试着让北京相信一旦朝鲜不让步,美国将毫不犹豫对用军事手段单方面地去解决问题。
 
 


\item “六方会谈”实际上已经从“对话的性质”转变成了“对抗的性质”\\
\label{sec-1_1_2}%
但是,就如我们在早前的《东方环球时事》中所说的那样,在布什先后于去年4月和7月份,派“极其强硬”的切尼、与切尼相比相对温和的赖斯相继访华,欲以一个毫无保障的“维持台海现状去换取北京放弃朝鲜”。

    事实证明,在北京“不再相信”台海有和平解决的可能性之后,确信美国霸权一天不除,台海就必有一战之后,自然也就一口拒绝切尼和赖斯的带去的 “以美国承诺维护台海现状的承诺、去换取北京放弃对朝鲜的军事保护”的建议。从根本上讲,切尼和赖斯这两位重量级人物的访华“相继失败”,实际上就是中国正式拒绝放弃朝鲜的必然结果。

    从而从那开始,从赖斯离开北京的那天开始,也就是从去年7月份之后,朝核问题的和平解决进程、实际上就彻底停顿下来。在东方评论员看来,“六方会谈”实际上也就从“对话的性质”转变成了“对抗的性质”,这就是为什么布什连任前朝鲜半岛局势“一转直下”的原因之所在。


\item 朝鲜何以敢与华盛顿叫板?\\
\label{sec-1_1_3}%
在我们看来,朝鲜之所以敢与华盛顿叫板,不仅是因为它是挟在中、俄、美和日本、韩国间的战略要地,更重要的是、它同时在地理和政治上“紧紧地背靠”着中国和俄罗斯这两个美国“不能拿对手怎么样”的大国,这与美国“敢说打就打”的伊拉克“没有保护人的情况”截然不同。

    东方评论员认为,尽管朝鲜远远没有伊拉克、伊朗那样的石油资源,然而
    \red{朝鲜政权的“稳定与否”直接关系到诸大国们的“全球战略的主动与否”。h}

    特别是对中美两国,由于本来就有个台海问题没有解决,“中美”国家利益更是在全球范围内的战略要地\footnote{全球战略梛有哪些?中东 }“频繁交手”、比如说中东,因此,在“中美两国”看来,营造出一个“对自己有利的”东亚形势至关重要,如此一来,在台湾问题上彼此都不敢轻易有大动作的情况下,各自在朝鲜半岛上所具有的地缘利益也就日显突出。

    首席评论员指出,从春节初一以来,我们从这段喧闹的日子里,可以清楚地看到,尽管中美两国都在东亚“努力营造一个对自己有利的东亚形势”。然而,不论是从地理位置,还是从国家战略、甚至是从日本侵略亚洲的历史来看,日本对朝鲜半岛战略价值的重要性更是“颇有心得”。

    也正因如此,在东方评论员看来,“朝核问题”或者说“朝鲜半岛紧张与缓和”已经成了大国们手中一张牵一发而可以动全身的“大牌”,在我们看来,“这张牌”不仅北京在打、华盛顿也想打、有意思的是,日本人也想挤进玩一把。

    在一段有关日本与韩国争夺独岛最新进展的相关报道之后,我们将继续这个话题,看看日本人是怎么玩这张牌的。


\end{itemize} % ends low level

\item 韩议员反日情绪高涨呼吁立刻驱逐日驻韩大使\\
\label{sec-1_2}%
汉城消息】据韩国联合通讯社报道,韩国政党进一步加大了对日本驻韩大使高野纪元就韩国独岛发表的有争议的言论的批评力度。一些议员甚至要求立刻驱逐高野纪元。

   日本驻韩国大使高野纪元23日称独岛(日本称竹岛)是日本领土。这在韩国国内引发了强烈反响。国会文化和旅游业委员会今天举行了全体会议,委员会成员纷纷发表了反日言论。委员会主席强调指出,韩国还没有从日本殖民主义的“阴影”中走出来。

   独岛位于韩国郁陵岛东面约70公里处,是由被称为“东岛”和“西岛”的两块岩石组成的岛屿。韩日两国都宣称对此岛拥有主权,目前韩国实际控制这一岛屿。


\item 朝核困境无妨援助韩开运六千多吨援朝玉米\\
\label{sec-1_3}%
中新网2月28日电 据法新社报道,韩国官员27日宣布,尽管目前朝核问题陷入困境,但韩国仍将履行去年的承诺,为朝鲜提供最后“一船”食品援助货物。

   这批货物包括大约6500吨玉米,这是韩国去年承诺向朝鲜提供的“10万吨玉米援助”的最后一部分,它们将在3月2日或3日运抵朝鲜西部港口。

   “韩国仍然保证将会继续向朝鲜提供人道援助,以便促进它与朝鲜之间的合作。”一位未透露姓名的韩国官员说。

   联合国粮农组织上月呼吁,国际社会应向朝鲜提供更多援助,称目前朝鲜的粮食缺口达到了50万吨(价值2亿2千万美元)。

   【时事点评】就日本这次主动挑起“独岛争端”,东方评论员也通过几天的跟踪分析给出了大量的分析,总而言之,我们的观点是清楚的,那就是,这是大年初一日本挑起钓鱼岛争端、“美日”以“擦边球”的方式抛出涉及台湾问题的“安全共同声明”、中国暂时在东亚核军备竞赛的问题上“引而未发”的继续。

   我们注意到,连日来,这次由日本岛根县议会表示将强行制定“竹岛之日(韩国称独岛)”条例案引起的领土争端,在韩国国内爆发了一连串的抗议活动。由于今年是韩国从日本殖民统治下光复60周年,又是韩日建交40周年,同时还是“韩日友谊年”。
 
\begin{itemize}

\item 充分地“展露”了小泉纯一郎的“狐狸尾巴”\\
\label{sec-1_3_1}%
在东方评论员看来,日本中央政府在整个过程中的态度“耐人寻味”,开始,它借口很难干涉地方议会事务,声称不介入此事。然而,就在日本政府做此声明、且在韩国方面提出正式抗议之后,日本驻韩大使居然在汉城特意举行的外国记者招待会上、公然强硬地声称“竹岛是日本领土”。

    与此同时,日本另一位外交官员也坚定地表示:“维护对竹岛的拥有权远比韩日关系重要,具有重要的价值。”我们注意到,这名日本官员还强调:“政府解决竹岛问题的原则,是在不影响韩日关系的基础上,维护领土主权。”

    显然,如果日本政府不想看到事情不可收拾的话,那么,就算是日本中央政府真的“很难干涉地方议会事务”,那么,日本政府总可以让它的驻韩国大使不要火上浇油”吧。

    在东方评论员看来,日本驻韩国大使在这个时候的“火上浇油”,可以说充分地“展露”了小泉纯一郎的“狐狸尾巴”。
 

\item 小泉政府似乎是在有意在给东北亚局势制造混乱\\
\label{sec-1_3_2}%
与此同时,我们注意到,对此,法国一家国际战略周刊就发表文章认为:从日本多个外交官的“强硬态度”来看,小泉政府似乎是在有意在给东北亚局势制造混乱。

    东方评论员认为,这种观点可以说是“一语中的”。东亚的形势是:一方面,日本在挑起钓鱼岛主权之争后,在答应华盛顿愿意将台湾问题纳入“美日安全共同声明”的同时,却没有将自己已经明确的“中国威胁论”推销出去,更孬提要华盛顿在钓鱼岛主权问题给出个让日本高兴的态度来。
 
 

\item 在中国以“半岛无核化”替代“朝核宣布有核”的“斡旋”下、朝鲜宣布“有条件地”返回“六方会谈”\\
\label{sec-1_3_3}%
而另一方面,则是朝鲜在宣布有核武器之后,由于有中国的强硬支持,再加上欧洲和俄罗斯都在那里“不痛不痒”地、“象是履行手续一般”批评朝鲜“做了一件错事”之后,就没了下文。

    结果,在华盛顿后退一步、打了一记“擦边球”之后,朝鲜方面并没有象传说中的那样,进行打击范围涵盖日本的中程导弹试射,也没有做什么进一步刺激日本的事情,反观北京,北京一方面继续“拒绝指责”朝鲜“宣布有核”、并有意“转移概念”、以敦促“各方都要表现出足够的诚意、通过谈判实现半岛无核化”的方式来“重新给朝核问题定性”,结果,在中国的这番以“半岛无核化”替代“朝核问题”的“斡旋”下、朝鲜立刻宣布“有条件地”返回“六方会谈”。
 
 

\item 这种安排表明北京“暂时还不打算”将“东亚核竞赛\protect\footnote{东亚核竞赛? }”这张“牌”立刻就给抛出来\\
\label{sec-1_3_4}%
显然,这种安排表明北京“暂时还不打算”将“东亚核竞赛”这张、日本人非常渴望见到的“牌”立刻就给抛出来。

    请大家注意这么一点,即,直到现在为止,在中国外交部的正式发言、或者中国领导人的谈话中,我们始终没有听到“朝核问题”这个词。

    有意思的是,一说到朝鲜半岛局势,北京方面的口径非常统一,那就是,要么用“半岛无核化”这个复杂的名词、要么干脆简化到“核问题”,总之,是在刻意地以这两种说法替代“朝核问题”这个词,东方评论员认为,北京“这样坚持”是有着深层的战略考虑的。
 

\item 恐怕不会再有人天真地相信:“美国”和“中国”是“清白无辜”的了\\
\label{sec-1_3_5}%
在我们看来,大年初一,日本将钓鱼岛上的灯塔“收归国有”,大年初二,朝鲜“宣布有核武器”,这两件事间的“关联性”到了今天,恐怕不会再有人天真地相信:“美国”和“中国”是清白无辜的了,东方评论员认为,不论日本和朝鲜是否愿意当华盛顿和北京的棋子,然而,就事情的过程和结果来看,日本和朝鲜在追求自己国家利益的同时,的确起到了“棋子”的作用。
 

\item 北京通过朝鲜“宣布有核武器”地继续给美国和日本的战略协调制造障碍\\
\label{sec-1_3_6}%
就华盛顿而言,日本在台湾问题上采取两面讨好的骑墙策略,实际上也是美日同盟背后的一大隐忧。如何化解这种隐忧,是华盛顿“一直在努力着的事业”。因此,这场较量下来,美国得到的不少,最大的战果就是那份谋划了多年、意在破坏“中日关系”基石的、涉及台湾的“美日安保指针”,初步达成了其着眼于中东、调整东亚安全格局的战略目的;

    对北京而言,由于日本帮着美国破坏了“中、日两国”牵制美国势力的“中日关系”,显然,北京的亚洲战略的确受损不轻,在台湾问题上也受损不轻。尽管日本在台海出现战火时,是否真会和美国一道协防还是个疑问,但是,
    \red{就目前的局势而言,台湾问题事实上已经由“中美对峙”变成了中国和“美日”对峙的局面。}

    然而,北京在台湾问题上面对更大压力的同时,显然也在朝鲜半岛上、通过朝鲜“宣布有核武器”这一点、在继续给美国和日本施加压力、继续给美国和日本的战略协调制造障碍。
 

\item 北京现在的策略似乎是在继续以“朝鲜半岛有核”这一事实为出发点,继续要求华盛顿做出选择\\
\label{sec-1_3_7}%
中国外交部发言人的态度非常清楚,第一,中国不认为“朝鲜半岛无核化”仅仅是特指的“朝核问题”,第二,由于“有核”是朝鲜官方的正式声明,因此“朝鲜半岛的确有核”已经是事实,但是,中国显然已经将其定性为“现有基础”,这就是说,“朝鲜半岛无核化”必须在“现有基础”上、以谈判的方式和平解决,显然,“无核化的范围”不仅包括朝鲜、也应该包括韩国、甚至直指“驻韩美军”有无核武器的问题。

    由于“美日安全共同声明”将台湾纳入其中已经是事实,华盛顿通过一记“擦边球”避免了直接选择“东亚核竞赛”这一结果,因此,在东方评论员看来,北京现在的策略似乎是在继续以“朝鲜半岛有核”这一事实为出发点,要求华盛顿在朝鲜准备正式、公开进行核武装、继而有可能引发东亚核竞赛、和“全面缓和”朝鲜半岛局势这两种结果中做出选择。

    在我们看来,朝鲜公开、正式进行核武装显然是华盛顿“绝不容忍”的,那样一来,日本和韩国也必将尝试发动国内社会,强烈要求跨入核门槛,从而出现核军备竞赛,从而对美国的“亚洲军事存在”产生极大的冲击。
 

\item 美国的战略家们也准备了一套替代“东亚核竞赛”的应付方案\\
\label{sec-1_3_8}%
同时,东方评论员也注意到,在朝鲜宣布有核之后,美国的战略家们也准备了一套替代“东亚核竞赛”的应付方案,那就是,美国也可能将在东亚可能实行更加强硬的军事存在,美国加强南韩和日本的军事力量。可问题是,在朝鲜正式跨入核门槛、拥有战略核打击力量之后,而美国仍然拒绝日本和韩国进入核门槛进行自卫的同时,日本和韩国的政治人物一定会扇动社会、对华盛顿施加压力,在这种情况下,日本和韩国社会、还会愿意缯加美国增强军事力量、继续接受美国的控制吗?

    当然,我们也注意到,美国的战略家还推测出了一种“可怕的结果”:认为,如果中国不和美国一道扭转局势,那么东北亚地区、包括日本,台湾,可能会感到更不安全,因而日本和台湾也可能想追求发展自己的核武器项目。
 

\item 以台湾可能发展核武威胁中国,根本就是在“拿台湾宣布独立”威胁中国\\
\label{sec-1_3_9}%
显然,这份名单里面没有韩国, 只有让中国头痛的日本,另外还加了个让大陆“无从着力”的台湾,东方评论员认为,其威胁北京的意图是一目了然,因为在一般的不具国际经验的人看来,韩国有核武器对美国不是好事,而只要一提日本和台湾核武装似乎就可以让北京头痛。

    东方评论员认为,我们无意去指责抛出这种观点的美国专家专业水平,因为一种经不起推敲的观点,不一定是因为持有者的“专业水平不高”,在我们看来,很多时候,在“别有用心”的情况下,似乎更容易产生这样低级错误。

    然而,东方评论员认为,如果台湾也敢发展自己的核武器项目的话,那事情反而变得简单了,因为在大陆订下的动武条件、除了前面所说的原则性的 “动武三前提”之外,还有一条中美心照不宣的硬条件,那就是“如果台湾追求发展自己的核武器项目等大规模杀伤性武器的话,大陆必然会武力解决台湾问题”。

    因此,只要美国还不想同中国摊牌,由于台湾问题和朝鲜问题不一样,是中国没有任何弹性的主权问题,因此,“台湾发展核武器的事”本身就赞同于 “台独”,在东方评论员看来,以台湾可能想发展核武器来威胁中国的主意,根本就是在“拿台湾宣布独立”来威胁中国,当然,如果美国有这个能力摊牌,自然可以用这一套来威胁,问题是,美国还没有这个能力。

    事实上,就在美国战略学家献计献策的时候,从朝鲜宣布“有自卫用的核武器”那一刻起,就一直保持克制的华盛顿,今天终于愿意开出条件来、“请”朝鲜回到谈判桌上。最新的消息是,朝鲜将会在6月重返“六方会谈”的谈判桌,并会在10月与美国签署一项条约。

    到此为止,东方评论员认为,朝鲜在中国的庇护下,不仅成功地迫使华盛顿做出让步,而且谈判的基础,也被北京修改成了“朝鲜有核”的“现有基础”。
 
 

\item 朝鲜局势有缓和的信号,这不能不让“欲乱中牟利”的日本焦急\\
\label{sec-1_3_10}%
反观日本:日本在主动于钓鱼岛主权问题上挑畔中国、并满足美国的要求、将自己放到了与中国公开对立的处境下之后,
    \red{日本在自己最关心的几个问题上}
    ,如,在确立“中国”为“美日同盟”的军事威胁的问题上、在“美日军事同盟”是否涵盖钓鱼岛的问题上,在引发“东亚核竞赛”的问题上,可以说是一无所得、这种结果似乎让小泉纯一郎冲着华盛顿、说了些日本领导人不应该说的一些话,比如,“将台湾纳入美日安保声明是华盛顿的意思”这一类的“外交忌语”,不论是出于何种原因,小泉的这种口不择言,让国际社会产生了“小泉被布什出卖了”的感觉。

    我们认为,在日本方面挑起中日钓鱼岛争端之后,在国际社会认为“小泉被布什出卖了”之后,不仅在日本所有可以用来实现“正常国家”的手段中、属于“最简便、高效的”---“东亚核竞赛”没有一丝迹象,反而朝鲜局势有缓和的信号,这不能不让“欲乱中牟利”的日本焦急。

    这一点,同韩国声称:尽管目前朝核问题陷入困境,但韩国仍将履行去年的承诺,为朝鲜提供最后“一船”食品援助货物、所表现出的有意为缓和半岛局势、营造气氛所做的努力、形成了“鲜明的对比”。
 
 

\item 日本对东亚核竞赛仍然没有死心\\
\label{sec-1_3_11}%
东方评论员认为,自上世纪初至今,韩日两国在独岛(竹岛)问题上的较量就一直没有停歇过,而这次日本再次以这种不具弹性的领土主权问题、去主动挑畔另一个亚洲大国--韩国,其目的恐怕就在于挑斗各自国家的民族情绪。

    首席评论员指出,在日本挑衅中国、于钓鱼岛方向“搞事”、却在“三大问题上”暂时“都没有如愿”的情况下,这次又挑衅韩国、于“独岛”方向再次“搞事”,其主要目的恐怕就是想尝试催动“东亚核竞赛”这一步了,换而言之,日本对东亚核竞赛仍然没有死心。
 

\item 在之后的事态发展中,我们认为有一个地方非常值得注意\\
\label{sec-1_3_12}%
首席评论员认为,在之后的事态发展中,我们认为有一个地方非常值得注意,那就是,朝鲜是否会主动介入这场“独岛(竹岛)争端”,显然,在我们看来,如果“六方会谈”仍然没有进展、换句话说,如果华盛顿仍然拒绝对朝鲜做出实质性的让步,从而让朝鲜半岛局势彻底缓和下来的话,那么,就算是朝鲜不朝日本方向试射导弹去刺激日本社会,而只需象上次那样,也声称对独岛拥有主权,那么,剩下的事情就是日本保守政府\footnote{日本保守政府 }的事了。

    可以想像的是,一旦如此,日本政府一定会不遗余力地大做文章,象在钓鱼岛争端上所做过的那样,尽全力去催动国内民族意识。与中国争钓鱼岛主权,日本不占上风,因为中国是大国,是政治大国、经济大国、更是军事大国、是核国家,这已经让日本政治人物“悲情过一把了”。

    可以想像的是,如果与朝鲜或者韩国争独岛主权也不占上风的话,由于朝鲜不如日本大,韩国不如日本强、假如将高丽民族与日本争锋、渲染成“就是靠的核武器”的话,日本政治人物的“悲情”无疑也就将更上一层楼。

    在我们看来,不论日本人如何做这篇文章,起码有一点是少不了的,那就是,“中心思想”肯定是这样的:“一个没有核武器的日本,是无法保障自己的主权的”。事实上,从我们得到的最新资料中可以清楚地看到,在日本,已经有人开始这样做文章了。

    下面,是一篇译自日本一家主流媒体的评论,其立场显然是右倾的。在一起浏览该文之后,东方评论员将继续讨论美日在“朝鲜半岛无核化”问题上的战略矛盾。




\end{itemize} % ends low level
\end{itemize} % ends low level
\section{2005年03月14日 星期一}
\label{sec-2}


  
\begin{itemize}

\item 《反分裂国家法》获通过(全文)\\
\label{sec-2_1}%
【人大网站】 《反分裂国家法》2005年3月14日第十届全国人民代表大会第三次会议通过

   第一条 为了反对和遏制“台独”分裂势力分裂国家,促进祖国和平统一,维护台湾海峡地区和平稳定,维护国家主权和领土完整,维护中华民族的根本利益,根据宪法,制定本法。

   第二条 世界上只有一个中国,大陆和台湾同属一个中国,中国的主权和领土完整不容分割。维护国家主权和领土完整是包括台湾同胞在内的全中国人民的共同义务。

   台湾是中国的一部分。国家绝不允许“台独”分裂势力以任何名义、任何方式把台湾从中国分裂出去。

   第三条 台湾问题是中国内战的遗留问题。

   解决台湾问题,实现祖国统一,是中国的内部事务,不受任何外国势力的干涉。

   第四条 完成统一祖国的大业是包括台湾同胞在内的全中国人民的神圣职责。

   第五条 坚持一个中国原则,是实现祖国和平统一的基础。

   以和平方式实现祖国统一,最符合台湾海峡两岸同胞的根本利益。国家以最大的诚意,尽最大的努力,实现和平统一。

   国家和平统一后,台湾可以实行不同于大陆的制度,高度自治。

   第六条 国家采取下列措施,维护台湾海峡地区和平稳定,发展两岸关系:

   (一)鼓励和推动两岸人员往来,增进了解,增强互信;

   (二)鼓励和推动两岸经济交流与合作,直接通邮通航通商,密切两岸经济关系,互利互惠;

   (三)鼓励和推动两岸教育、科技、文化、卫生、体育交流,共同弘扬中华文化的优秀传统;

   (四)鼓励和推动两岸共同打击犯罪;

   (五)鼓励和推动有利于维护台湾海峡地区和平稳定、发展两岸关系的其他活动。

   国家依法保护台湾同胞的权利和利益。

   第七条 国家主张通过台湾海峡两岸平等的协商和谈判,实现和平统一。协商和谈判可以有步骤、分阶段进行,方式可以灵活多样。

   台湾海峡两岸可以就下列事项进行协商和谈判:

   (一)正式结束两岸敌对状态;

   (二)发展两岸关系的规划;

   (三)和平统一的步骤和安排;

   (四)台湾当局的政治地位;

   (五)台湾地区在国际上与其地位相适应的活动空间;

   (六)与实现和平统一有关的其他任何问题。

   第八条 “台独”分裂势力以任何名义、任何方式造成台湾从中国分裂出去的事实,或者发生将会导致台湾从中国分裂出去的重大事变,或者和平统一的可能性完全丧失,国家得采取非和平方式及其他必要措施,捍卫国家主权和领土完整。

   依照前款规定采取非和平方式及其他必要措施,由国务院、中央军事委员会决定和组织实施,并及时向全国人民代表大会常务委员会报告。

   第九条 依照本法规定采取非和平方式及其他必要措施并组织实施时,国家尽最大可能保护台湾平民和在台湾的外国人的生命财产安全和其他正当权益,减少损失;同时,国家依法保护台湾同胞在中国其他地区的权利和利益。

   第十条 本法自公布之日起施行。

   【时事点评】说实话,在拿到“反分裂国家法”的正式文本之后,东方时代的时事评论员们普遍感觉到了“有种震憾”,即,“反分裂国家法”并没有划出“极其清楚”的“红线”,这就是说,将来“台独势力”如果真的制造了“台独重大事变”,那么,对于什么是“台独重大事变”,中国人大将通过“解释反分裂法”、而不是以“对号入座”的方式去加以认定,那么,这种“释法”加以认定、和明确划定“红线”之间到底有何不同呢?

\begin{itemize}

\item 《反分裂国家法》的核心要义在什么地方\\
\label{sec-2_1_1}%
在我们一步给出“具体观点”之前,我们还是先来看看这部已经生效了的《反分裂国家法》的核心要义在什么地方。纵观全文,不难看出,“反分裂法”的核心在于第二条,第八条。

    在东方评论员看来,第二条的重要性在三点:

    第一,在于它将“世界上只有一个中国,大陆和台湾同属一个中国、台湾是中国的一部分”这种政策性宣示,从法律上加以明确了,也就是说,正式在法律上确认了两岸关系的现状就是大陆和台湾都属于一个中国的一部分。

    第二,今后,任何与中国正式外交关系的国家,由于事实都是在确认中华人民共和国是中国的唯一合法政府的前提下,与中国建立外交关系的。所以,由于台湾是中国的一部分,在涉及台湾问题时,这些“建交国”的任何行为都有必要不与《反分裂国家法》相冲突、而中国在执行《反分裂国家法》处理台湾事务的过程中,在寻求相关国家的帮助的问题上,比如协助抓捕,并引渡“台独罪犯”、切断“台独”势力的经济来源等等、对中国和被中国要求提供帮助的建交国而言, “反独”也就“都有了”法律依据。

    第三,最后,也是最重要的,这是今后人大“释法”、鉴别是否满足“三条件”、判别是否需要采取“非和平方式”的去维护中国统一的法理根据。
 

\item 一个“动用”“非和平方式”的条件“变更”引起了我们注意\\
\label{sec-2_1_2}%
其次,就是第八条,原文是:“台独”分裂势力以任何名义、任何方式造成台湾从中国分裂出去的事实,或者发生将会导致台湾从中国分裂出去的重大事变,或者和平统一的可能性完全丧失,国家得采取非和平方式及其他必要措施,捍卫国家主权和领土完整。

    首席评论员指出,第八条中的一个“动用”“非和平方式”的条件“变更”引起了我们注意,既“任何方式造成台湾从中国分裂出去的事实,或者发生将会导致台湾从中国分裂出去的重大事变,或者和平统一的可能性完全丧失,”

    根据与北京事先披露的“草案”进行比较,很明显,启动“非和平手段”的条件还是“三个条件”,但是, 我们不难发现,最后一条已经由“和平统一的条件完全丧失”悄然换成了“和平统一的可能性完全丧失”。

    东方评论员认为,通过这一条件的修正,不难看出“修正后”的“正式文本”,较之“草案”相比,更加体现出了一种特性,那就是“最大的弹性”。
 

\item 人大获取了一种“主观解释的便利”\\
\label{sec-2_1_3}%
在我们看来 ,这种“最大的弹性”是指将来一旦人大有必要“释法”界定中央军委、国务院对台启用“非和平方式”“是否适当”时,更加具有“主观性”和“模糊性”。

    显然,由于《反分裂国家法》已经正式通过、且即日生效,因此,自今天起,全国人大实际上“已经正式授权”中央军委、国务院对“台独事实”或者“台独生大事变”有“当机决断”、“事后请批”的权力。

    所以,在东方评论员看来,这本质上是让中央军委、国务院今后有更大的主动性和灵活性、视国际形势的变化去“便宜行事”,而人大也可以利用这种“模糊性”来为启动”非和平方式”的正确与否,获取了一种“主观解释的便利”。
 

\item “释法界定”去取代“具体划红线”的方式,无疑更富有“进攻性”\\
\label{sec-2_1_4}%
东方评论员认为,由于《反分裂国家法》是以一种“主观判断的”“释法界定”去取代“具体划红线”的方式,来判断是否满足启动“非和平方式”的“三条件”,在我们看来,在当前这种国际局势跌宕起伏,变幻难测的形势下,无疑更富有“进攻性”。

    尽管在之前的《东方时事解读》中,我们一再强调过“反分裂法”的立法过程也许是个“有趣的过程”,我们也一再强调过“反分裂法”的内容确定和立法进程都要“超越”台湾问题、而着眼于世界格局、紧盯着中东局势,而在内容上加以“紧松”、在进程上控制“快慢”。

    然而,我们也一直坚持认为,在具体条文上,有必要具体划定几条主要的“红线”、以防止“台独”和支持“台独”的国际势力“立刻”就进行冒险、伺机挑畔“反分裂国家法”的权威。
 

\item “反分裂法”的确是充分考虑到了内容的“紧松”问题\\
\label{sec-2_1_5}%
事实说明,“反分裂法”的确是充分考虑到了内容的“紧松”问题,在东方评论员看来,这种“充分考虑”的结果就是将以“解释反分裂法”“第八条”的方式,去灵活界定“是否启动”“非和平方式”。

    可以这样说,这种“将释法第八条”的方式,可以说是根据国家的意志、和国际形势的具体情况,一旦北京认定有必要抢在美国决定打“台独牌”之前,提前解决台湾问题、而需要从“紧”,可以说就可以立刻启动台海战争。

    在我们看来,陈水扁稍早宣称的“台湾是主权国家、台湾主权的变更需要台湾2300万人决定”的说法,就可以解释成“和平统一的可能性完全丧失”,从而立刻启动“非和平方式”。
 
 

\item 人大“似乎有意时刻准备着”去“解释第八条”\\
\label{sec-2_1_6}%
然而,首席评论员指出,通过人大“似乎有意时刻准备着”去“解释第八条”这点来看,如果国际形势的发展表明:华盛顿有可能在自己意识到中东战略彻底破产、反手大打“台独牌”、以冲击中国的方式、打乱“中欧俄”的战略协调、进一步压缩中日之间的战略回旋空间、从而去提前策应其全球部署;

    在这种情况下,如果北京认为暂不宜启动“非和平方式”似乎更合适的话,可以想像的是,人大将从“松”的角度去解释“第八条”。不要忘记了,“非和平方式”中除了有极端的武力攻打之外、还有经济封锁、军事封锁之类的、相对缓和的手段可供选择。
 

\item 北京也有必要视情况从“紧”去“解释第八条”\\
\label{sec-2_1_7}%
同时,我们还应该警惕,美国或者在其中东战略全部得手之后,在完成其全球布置、完成针对中国的战略布置之后,那么,华盛顿也可能反手大打“台独牌”、利用“台独”来打断中国的正常建设、给中国制造巨大的战略困难。

    显然,为了避免这种被动的局面发生,北京也有必要视情况从“紧”去“解释第八条”。由此可见,中国立法部门和执法部门、如何去理解、解读这个“第八条”端的是要视世界局势发展而定,才能最大限度地保障中国的核心利益。
 

\item 由于没有具体的“红线”,《反分裂法》将“紧松”问题发挥到极致的同时,也是有负面作用\\
\label{sec-2_1_8}%
然而,我们在看到《反分裂法》将“紧松”问题发挥到极致的同时,也是有负面作用的:由于没有具体的“红线”,那么,台独和支持台独的国际势力,特别是华盛顿,很可能会选择一个时机去挑畔、测试“反分裂法”的权威,为极可能启动的“两岸和谈”、甚至中东问题的谈判制造足够筹码。
 

\item 尽管吕秀莲的“这一总结”充满了“怨恨”、然而多少还显露了几分“政治素质”\\
\label{sec-2_1_9}%
事实上,针对这种人大的“主观解释上的便利”,东方评论员注意到,在美国休斯顿过境的台湾“副总统”吕秀莲就表示出强烈不满,她说:《反分裂法》条文虽然少一条,但却一点都没有让步,并指责“中国一手掌握立法、诠释、制裁权”,还认为:条文内容全是“反独促统”的政治声明,强迫台湾接受“一中原则”。

    在我们看来,尽管吕秀莲的“这一总结”充满了“怨恨”、然而,撇去感情因素,客观地讲,她的这几句话多少还显露了几分“政治素质”。

    的确,东方评论员认为,由于北京不仅掌握着“立法权”、更重要的是有“诠释权”和“制裁权”,因此,《反分裂国家法》通过并立即生效之后,也就立刻引起了巨大的反响。

    下面,我们先通过几则消息,首先看看台湾内部的反应,之后,我们将继续这一话题。

\end{itemize} % ends low level

\item 陆委会回应反分裂法「谴责」大陆要求道歉\\
\label{sec-2_2}%
台湾消息】据台湾媒体报道,针对中国大陆「反分裂法」,陆委会14日代表台湾当局发表声明,「谴责」中国大陆的作法,应该为此向台湾人民「忏悔」、「道歉「,同时也呼吁其它国家共同给予谴责。

   陆委会主委吴钊燮发表四点声明,声称「中华民国」主权属于台湾2300万人民,绝不容许中国大陆藉任何手段「侵犯」,任何改变都只有台湾人民有权决定。吴钊燮表示该法提供「以非和平方式处理台海问题」的法理基础,为解放军开出「并吞」台湾的「空白支票」,并要大陆向台湾人民「忏悔」。

   这份声明不但「谴责」中国大陆,更明白要求中国大陆必须道歉,否则台湾方面也不排除会采取必要措施,降低对两岸关系的不利影响。

   声明再次鼓吹所谓「台湾不隶属于中华人民共和国,中华民国与中华人民共和国共存于世且互不隶属」,并假借民意说这是「全体台湾人民」明确的共同主张;是台海长久以来的「现状」。声明还污蔑大陆是「东亚地区动荡不安的最主要根源」。


\item 北京通过反分裂法马英九提“公开信”抗议\\
\label{sec-2_3}%
【台湾消息】据台湾媒体报道,北京14日上午通过反分裂国家法,
   {台北市长马英九}
   提公开信,表达台湾大多数人民都希望维持“中华民国”现状,但反分裂法漠视台湾主流民意,既无必要也不明智,反而激起反感,对两岸关系发展带来不必要的阴影。马英九下午还将邀集泛蓝的县市首长一起表示抗议。

   对于反分裂法14日得以通过,马英九上午提出“公开信”抗议,下午则要邀集泛蓝执政的县市首长公开抗议,并举行国际记者会。

   马英九所提出的公开信,一开始即指出,“坚持和平,对等协商-为抗议‘反分裂国家’致国际社会的公开信。”据信中指出,北京通过针对台湾而制订的反分裂国家法,这一举动,引起台湾人民强烈的反感与国际社会极大的关注,也对两岸现状投下重大变数。我们身为“中华民国”地方政府首长,为了向国际社会表达绝大多数台湾基层民众的不满与抗议,特别举行国际记者会,对国际媒体发表这封公开信。

   公开信内容指出:我们认为:大陆当局应该清楚认知,“中华民国”自1912年“开国”以来,就是“主权独立”的“国家”,迄今并未改变。事实上,台湾地区大多数人民都希望维持“中华民国”现状,少数主张“台湾独立”与“正名制宪”的人士,并不能代表台湾的主流民意。何况近月来,陈水扁曾在国际场合明白宣示“正名制宪”之不可能;“行政院”谢长廷院长也指出,政府必须遵守“一中宪法”。

   另外,台湾也有禁止主张分裂“国土”的法律。因此,大陆当局以台湾少数人的主张作为理由来制订“反分裂国家法”,明显漠视当前台湾大多数人主张维持“中华民国”现状的主流民意,既无必要,也不明智,反而激起大多数台湾人民的反感,对两岸关系的发展带来不必要的阴影。事实上,台湾坚定维持“中华民国”现状,就是解决两岸问题的关键,也是台湾朝野各党派的最大公约数。

   海峡两岸自1949年起即处于分治状态,大陆当局从未统治过台湾,目前台湾治权也不及于大陆,因此双方的政治争议应该在维持现状-也就是“大陆不武、台湾不独”—的基础上,透过双方对等协商的和平方式来解决。但今日大陆当局以单方制定国内法的方式,意图以“内部”问题窄化两岸争端,并明示可能以非和平方式处理两岸问题,对两岸的互动投下重大变数,实非必要与明智之举。对此,我们必须严正表达我们的不满与抗议。

   台海和平是东亚区域稳定与安全的关键因素,有赖海峡两岸政府真正展现追求和平的决心与具体行动。值此两岸甫因今年初春节包机合作良好而展现和解善意之际,我们呼吁两岸仍能珍惜此一难得契机,以客观冷静的态度面对问题、解决问题。“反分裂国家法”的制定固然令人感到遗憾,但我们并不愿看到两岸因此再度走向激情对抗,仍然主张尽快恢复对话,透过对等协商解决争议,以维护台湾民众的利益并符合国际社会对两岸和平的期望。

   我们希望国际社会充分了解台湾主流民意的现状,以及解决两岸问题的关键,鼓励并协助两岸双方在维持“中华民国”现状的基础上,展开和平的对等协商,为两岸人民谋求福祉,也为区域稳定作出贡献。我们也呼吁大陆当局放弃单方面以非和平手段解决两岸问题的作法。最后,我们更期待执政当局能够以更理性务实的态度,以促成两岸和平为目标,更积极推动两岸对话与协商。

   【时事点评】我们注意到,《反分裂国家法》可以说是正式通过了,今天,可以说直到现在为止,我们并没有看到陈水扁公开跳出来说点什么,就是那个不久前在前台”跳得最高”“叫得最响”的台“行政院长”,在事先已经安排有“记者招待会”将表明立场的情况下,最后竟然也是“杳无意讯”、让媒体记者白等一场。
 
  我们看到的是,就在台湾的“总统”、“行政院长”都因种种原因没有露面的情况下,却推出了这个差点被陈水扁“整编掉”的“陆委会”、来代表台湾当局发表声明,表明态度。
 
\begin{itemize}

\item 吴钊燮是闭口不提“和平条文”,死死地咬住“第八条”\\
\label{sec-2_3_1}%
从新闻中可以清楚地了解到,陆委会主委吴钊燮在所发表的“四点声明”中,充满了对“反分裂法”的攻击之辞,其手法很简单,就是“断章取义”。

    我们注意到,吴钊燮是闭口不提“反分裂国家法”中的“和平条文”,死死地咬在有关启动“非和平方式”的“第八条”上,一口咬定“该法提供'以非和平方式处理台海问题'的法理基础,为解放军开出'并吞'台湾的'空白支票',并要大陆向台湾人民'忏悔'、'道歉'”。
 

\item 如果台独势力继续搞台独,那么,“第八条”也的确就是张“空白支票”\\
\label{sec-2_3_2}%
在东方评论员看来,就如我们之前所说的那样,如果台独势力继续搞台独,那么,“第八条”也的确就是张“空白支票”。然而,问题是,面对这张 “空白支票”,台独的“大头目”、比如陈水扁、谢长庭为何不现身抗争、而派出一位吴钊燮在那里威胁说什么“台湾方面也不排除会采取必要措施,降低对两岸关系的不利影响”。
 

\item 吴钊燮的这句话就“说大了”\\
\label{sec-2_3_3}%
事实上,在我们看来,吴钊燮的这句话就“说大了”:要知道,如果“台独”势力还能做得了主去“采取必要措施”,有决心去降低“两岸关系”的话,那么,我们又怎么会有机会看到那个、不仅对陈水扁、对民进党、甚至是对整个“台独”和支持“台独”国际势力都是“只失不得”的“春节包机”呢?



\item 马英九的表态,引得时事评论员们“有话要说”\\
\label{sec-2_3_4}%
显然,“春节包机”是华盛顿逼着陈水扁放飞的,因此,
    \red{在台湾如何反应的问题上,真正做得了主的是华盛顿。}
    因此,在我们没有看到美国方面的正式反应之前 ,东方时代的评论员们也压根就没有指望陈水扁能针对这部“反分裂国家法”、“敞开心扉”地说出个“一二三”来。

    因此,东方评论员认为,对陈水扁通过吴钊燮亮明的态度,我们“没有往心里去”,相较而言,倒是这个国民党的政治明星马英九的表态,引得时事评论员们“有话要说”。

    我们注意到,台北市长马英九在公开信中,开口闭口地大提什么“台湾大多数人民都希望维持“中华民国”现状”,大提什么“反分裂法漠视台湾主流民意”,是“既无必要也不明智,反而激起反感”。

    在东方评论员看来,就如之前所说的那样,我们早已经将马英九和宋楚瑜一样,被归之为“
    \red{ 隐性台独}
    ”,并定性为“台独”的一大隐忧。
 

\item 马英九指责“大陆当局以台湾少数人的主张作为理由来制订“反分裂国家法””的意思何在?\\
\label{sec-2_3_5}%
请大家注意这么一段:原文是,公开信内容指出:我们认为:大陆当局应该清楚认知,“中华民国”自1912年“开国”以来,就是“主权独立”的 “国家”,迄今并未改变。事实上,台湾地区大多数人民都希望维持“中华民国”现状,少数主张“台湾独立”与“正名制宪”的人士,并不能代表台湾的主流民意。何况近月来,陈水扁曾在国际场合明白宣示“正名制宪”之不可能;“行政院”谢长廷院长也指出,政府必须遵守“一中宪法”。

    另外,台湾也有禁止主张分裂“国土”的法律。因此,大陆当局以台湾少数人的主张作为理由来制订“反分裂国家法”,明显漠视当前台湾大多数人主张维持“中华民国”现状的主流民意,既无必要,也不明智。

    对这一段,东方评论员想分析的是,马英九指责“大陆当局以台湾少数人的主张作为理由来制订“反分裂国家法””的意思何在?
 

\item 我们从这位马英九的身上,再一次看到了宋楚瑜的影子\\
\label{sec-2_3_6}%
显然,他强调“近月来,陈水扁曾在国际场合明白宣示“正名制宪”之不可能;“行政院”谢长廷院长也指出,政府必须遵守“一中宪法””这一情况在前,不难看出,这位国民党副主席、有意竞选下一任国民党主席和台湾“总统”的马英九,已经将陈水扁当局归之为“ 主张维持“中华民国”现状的主流民意”的范围内。

    这就是说,我们从这位马英九的身上,再一次看到了宋楚瑜的影子,至于马英九是否会成为宋楚瑜第二,什么时候会成为宋楚瑜第二,相信不久就会看到答案。

    东方评论员认为,在华盛顿的授意下、宋楚瑜已经和陈水扁打得火热,搞出了个“联合声明”和“十点共识”,那么,马英九或者是说国民党又准备和陈水扁一起搞出个什么花样出来呢?
 

\item “反分裂国家法”生效之后的、又一个斗争焦点\\
\label{sec-2_3_7}%
在我们看来,马英九和陈水扁,或者说是国民党和民进党、在美国的安排与压力下,能拿出来的花样之一,恐怕就是那个我们一直挂在嘴边的“中程协议\footnote{中程协议 }”了。

    然而,我们曾经说过,“中程协议”的条文不是主要问题,关键是前提条件,那就是“一中原则”,或者“九二共识“\footnote{”一中原则''、''九二共识'' }。因此,在我们看来,在华盛顿在全球战略被动的情况下,在在华盛顿想调整东亚布暑、以防止北京钻华盛顿的战略空子,排挤美国在亚洲利益的安排、还没有达成目的的情况下,如何拿到一个“名义上”可以保持台海几十年和平的“中程协议”,就成了“反分裂国家法”生效之后的、又一个斗争焦点。
 

\item 北京可以同意的是在“一中原则”下的“中程协议”\\
\label{sec-2_3_8}%
东方评论员认为,尽管“反分裂法”中没有明指“两岸和谈”必须是承认“一中原则”,但是,由于胡锦涛在4日的讲话中、于“四点意见”非常明确地列明了这一条件。

    因此,可以肯定的是,就目前而言,由于国际形势复杂多变、台海和平还是比战争要容易控制,因此,北京可以同意的是在“一中原则”下的“中程协议”;

    而对于台湾而言,由于“的反分裂国家法”的巨大压力,陈水扁、宋楚瑜已经在美国的主导下、搞了个“十点共识”、承诺必须遵守“一中宪法”、而马英九又在借指责北京之余、顺带为陈水扁、宋楚瑜承诺必须遵守“一中宪法”进行“背书”。因此,在东方评论员看来,台湾的目的也很清楚,那就是,
    \red{在华盛顿因其战略需要暂缓台独进程的时期内,台湾各政治势力的一个共识就是:“一中宪法”下的“中程协议”。}
 

\item 美国人要一份“中程协议”的最终目的很明确\\
\label{sec-2_3_9}%
至于美国人,其实最终的目的很明确,那就是一个“名义上”可以确保几十年的“中程协议”,至于是否是“一中原则”已经顾不得了。

    \red{只要有了这份“中程协议”那么,一来可以借自己是唯一能控制“台独进程”的力量这一点、来压中国在全球战略上做出让步、二来借“反分裂法”去约束台独,从而尽可能地去维持一个短暂的台海和平。}

    华盛顿则可借机全力推进其中东战略,从而全力推进其全球战略。第三,也是最重要的是,那就是,华盛顿一旦在中东得手、在东亚面置完毕,那么,就可能随时鼓动“台独”撕毁“中程协议”,让“台独”成为一个彻底消耗中国实力的强大工具。
 

\item 华盛顿可以随时鼓动“台独”撕毁“中程协议”,北京也可以视情况“主观解释第八条”\\
\label{sec-2_3_10}%
然而,现在“反分裂国家法”的正式文本却是一个以“解释第八条”为启动“非和平方式”的“弹性版本”,因此,华盛顿如想借“反分裂法”去量化“台独进程”,反倒显得没有了标准。

    也就是说,华盛顿可以随时鼓动“台独”撕毁“中程协议”,让“台独”成为一个彻底消耗中国实力的“强大工具”,而北京也可以视情况“主观解释第八条”,也可以随时启动“非和平方式”去提前解决台湾问题、从而在华盛顿的“台独时间表”之外,去提前解决“台独”,将“台独”变成美国的战略包袱,极大地干扰其全球战略计划。
 

\item “弹性版本”的“反分裂法”的通过,的确给了我们一种“震撼”,\\
\label{sec-2_3_11}%
事实上,在东方评论员看来,这个弹性版本的“反分裂法”的通过,的确给了我们一种“震撼”,那就是,中国政府在保持“弹性”启动“非和平方式”的同时,也意味着有极大的“决心和意志”去宣示“台海必有一战”、去面对“台海必有一战”。
 
 

\item 对中国而言,一个由自己挑选时机启动的台海战争,绝对比一个由美国挑选时机的台海战争、要来得容易得多\\
\label{sec-2_3_12}%
在这里,我们想引用温总理的一句话,他在回答一个美国记者的有关“反分裂法”的问题时说,''你可以翻开1861年贵国制定的两部反分裂法,不也是同样的内容吗?而且随后就发生了南北战争。''

    在东方评论员看来,不论是有意还是无意,这句话给人的感受很深:美国的两部反分裂法没有阻止南北战争,因此,中国的“反分裂法”很可能也阻止不了台海战争,然而,我们也有理由相信,如果“台湾必有一战”的话,那么,一个由中国自己挑选时机启动的、“主动统一中国”的台海战争,绝对比一个由美国准备好之后,再挑选时机、让台独挑起、北京被迫进行的镇压分裂势力的台海战争、要来得容易得多。

    下面,我们刚收到一则来自华盛顿的消息,在一起了解内容之后,东方评论员将就华盛顿对“反分裂法”通过之后的可能动作进行分析,并做为今天台湾部分的结束。





\end{itemize} % ends low level
\end{itemize} % ends low level
\section{2005年03月15日 星期二}
\label{sec-3}

  
\begin{itemize}

\item 外界高度关切台美间是否重建构水下监听系统\\
\label{sec-3_1}%
【台湾消息】台湾媒体报道称,由台“国防部”2年出版1期的国防报告书中,连续两期提及“水下监听系统”等用语,令外界高度关切台美间是否重启建构水下监听系统“龙睛计划\footnote{龙睛计划 }”?知情官员说,中科院确自上开计划获得“水中听音器”技术,惟随海军转向建构8艘柴电潜艇,可望获致主动反潜能量,水下监听将仅用以验证电讯情资。

   2002年度“国防”报告书(页284)首度指出,“国军”建立“电子侦测网”目标,在整合强网、大成等电讯信息,并结合无人遥控载具与“水下监侦系统”,藉联合监侦、识别、频谱管理机制,掌握敌我电磁频谱,支持全般作战。

   该年度“国防”报告书强调,为掌握空中、地面、“水下监侦系统”电讯情资,及敌我电磁之频谱,现正执行电磁频谱管理系统的研究。

   2004年度“国防”报告书168页表示,“国军”为反制潜艇威胁,持续进行线导鱼雷、复合感应水雷等水下武器关键性技术研发,包括:声纳浮标、舰装声纳系统、“水下监听验证系统”及反鱼雷诱标系统等,以提升反潜作战能力。

   “水下监侦系统”与“水下监听验证系统”差异何在?知情官员说,当年台、美合作“龙睛计划”失败,但中科院万象馆仍从合作过程中获得计划主要“水中听音器”相关技术。

   官员描述,这是一种能将电能转换成声波水中进行传播,或将声波转换成电子讯号加以分析应用的技术,是当时“SOSUS”的主要侦测器“水中听音器”制成技术,中科院可利用这项成果,执行水下声波、电讯的验证,与布放水下监听系统的“龙睛计划”大不相同。

   官员表示,未来海军若顺利获得8艘柴电潜艇,舰上声纳网结合“博胜案\footnote{博胜案 }”,自可组成主动式的反潜网,加上引进线导鱼雷、潜射导弹与水雷布放能量,中科院万象馆研发功能降低。

   【时事点评】据东方军事评论员介绍,台湾这个所谓的“水下监侦系统”已经吵吵嚷嚷多年了。事实上,台“国防部”现在接二连三地抛出这个“项目”,其用意却在新闻的最后一行,即,(台湾)官员表示:未来海军若顺利获得8艘柴电潜艇,舰上声纳网结合“博胜案”,自可组成主动式的反潜网。
 
\begin{itemize}

\item 我们又不能将这笔“军购案”仅从商业的角度去看待\\
\label{sec-3_1_1}%
显然,绕来绕去,又绕到了那“8艘柴电潜艇”的问题上,也就是说,台“国防部”这几天将这个一度“忌讳莫深”的“项目”拿出来“翻炒”,起码有一层意思,还是为了那笔“军购案”。然而,这笔能为美国军火商提供巨额商业利润的军售,我们又不能将这笔“军购案”仅从商业的角度去看待。
 

\item 台湾这笔几千亿新台币“军购案”离通过的时间不会太久\\
\label{sec-3_1_2}%
在东方评论员看来,就如我们之前所说的那样,由于华盛顿已经主导了台湾政治势力的整合,目前,陈水扁的民进党和宋楚瑜的亲民党已经被一纸“联合声明”、通过“中间路线”这个“政策名词”被紧紧地挷在了一起,因此,华盛顿一直压台湾通过、却被“国亲”主导的台湾立法会阻拦的“对台军售案”问题,事实上“已经不存在了”,可以肯定的是,台湾这笔几千亿新台币“军购案”离通过的时间不会太久。
 

\item 美日台之间的军事互动,其紧迫性也就立刻凸现出来了\\
\label{sec-3_1_3}%
在之前的点评中,我们已经说过,由于美日台无力阻止“反分裂法”的通过,因此,对台独以及支持台独的势力而言,“台独进程”如何调整现在已经日见清晰,
    \red{在东方评论员看来,这就是由“法理台独”为主、转向“武力拒统”为主的“台独政策”}
    。

    我们认为,在台独势力避开“反分裂法”的锋芒、而转走“武力拒统”的时候,如何加速美台、美日台之间的军事互动,其紧迫性也就立刻凸现出来了。

    在一段相关新闻后,东方时事评论员、军事评论员将一起就此问题进行展开。

    \href{http://www.dongfangtime.com}{《东方时代环球时事解读》}

\end{itemize} % ends low level

\item 台军汉光演习本周展开将与美日新联机推演\\
\label{sec-3_2}%
【台湾消息】据台媒报道,台军“汉光二十一号演习”兵棋推演将在本周展开,据了解,前来台湾指导“汉光二十一号演习”的美日台计算机兵棋联机的外军顾问已陆续抵达台湾,外军顾问团人数将首度破百。

   权威消息指出,为反制“反分裂国家法”,台军“汉光二十一号演习”兵棋推演将在本周展开,并且台军本周将借由“汉光”演习兵棋推演,进行美军太平洋总部联机兵棋推演,参与国有美、日及新加坡。

   台“国防部”经与相关单位协调后正式核定“汉光二十一号演习”将在中部地区展开,演习区域以彰滨工业区因应反突袭、反破坏、反恐等作为进行演练。

   据透露,台美双方已规划兵棋联机,预计今年美方将正式批准通过,由于美国已与日本联机,因此台、美、日未来都将透过太平洋总部联机兵棋推演。“国防部”目前已引进新的联合作战指挥系统,具备与他国联机的功能。

   军方高层指出,这一次的“汉光二十一号演习”预计将动用到清泉岗基地配合救援机队等进行演练作业,同时将大量运用后备军人体系进行城镇保卫作战,而彰滨地区在演习实兵行动中,将会出现包括海军陆战队登陆的“联兴演习”、陆军空降部队降落的“联云演习”及各种火炮射击等“联勇演习”。

   【时事点评】事实上,早在3月8日,就在《反分裂国家法(草案)》正式提交全国人大审议的当天,东方评论员就注意到,针对刚刚提请审议的《反分裂国家法(草案)》,台湾“国家安全会议”秘书长邱义仁专门主持召开了内部会议,称“反分裂法条文中所指‘非和平’手段,其实比‘动武’更加严厉”。
 
\begin{itemize}

\item “台独”想掌握“反分裂法的翻译权”,想得的确不错\\
\label{sec-3_2_1}%
值得一提的是,台湾当局声称:现在台湾要做的行动主要有两个:一是掌握“反分裂法的翻译权”,以便更好地向美日等国“告状”;二是要夸大大陆的“军事威胁”。

    在东方评论员看来,“台独”想掌握“反分裂法的翻译权”,想得的确不错,然而,这种“翻译权”在大陆的手中,显然,“翻译权”是要靠实力做保障的。至于如何夸大大陆的“军事威胁”,陈水扁当局是“行家”,何况旁边还是“美日”两个高参,当然清楚怎么做。

    于是乎,东方评论员立刻就看到,与此同时,台“国防部”也是配合默契、立刻抛出了2005年度重大演习训练规划表。
 

\item 陈水扁这一路演下来,分明是在向大陆展示其武力对抗《反分裂国家法》的姿态\\
\label{sec-3_2_2}%
根据这份规划表,台军今年仅是军演就安排了50次之多。这还不算,第二天,台军的动作就从备战换成了渲染大陆的“军事威胁”。也就是9日一大早,在与李敖于台“立法院”内争辩台湾是不是美国的“看门狗”的问题之后,台“国防部长”李杰仍然是如期“提交报告”,大肆渲染说大陆的军力是如何如何,怎样怎样。

    在东方评论员看来,陈水扁这一路演下来,分明是在向大陆展示其武力对抗《反分裂国家法》的姿态。

    对华盛顿而言,在“台独进程”被“反分裂法”彻底封堵之后,尽快在政治层面给两岸画出一个“和平缓和”的“和平前景”、以暂时稳住中国,并将对台政策由“法理台独”转向“武力拒统”也就具有了非常现实的意义。
 

\item 看美国是如何“玩转”“与台湾关系法”的\\
\label{sec-3_2_3}%
首席评论员指出,在之前,由于美国手中有一个“与台湾关系法”,一般认为,华盛顿就是凭借这个、有意在“协防台湾的问题上”“模糊其辞”、在它需要的时候,就做出一些相应的动作来,以让国际社会相信美国必然会“依法”介入台海的;

    同时,华盛顿还在它认为需要的时候,也可以任意变脸,做出一些相反的言行举止、好让国际社会相信华盛顿的介入“是有条件的”,也就是说,美国在决定“是否介入”之前,也要对“与台湾关系法”中的有关“协防台湾”的启动条件去加以“解释”。
 

\item 有一点足以让华盛顿的决策层感觉“非常难受”\\
\label{sec-3_2_4}%
然而,这都只是“以前的玩法”。东方评论员认为,在《反分裂国家法》已经生效的情况下,可以这样说,由于
    \red{《反分裂国家法》对中国政府具有法律约束力,并授以中央军委、国务院可以“见机启动战争”的权力,此外,对“台独事实”“台独重大事变”、以及“和平统一的可能性全部丧失”这些关键问题的解释权也操持在中国的立法机构手中。}

    因此,在东方评论员看来,有一点足以让华盛顿的决策层感觉“非常难受”,那就是:《反分裂国家法》的“正式文本”在表面上似乎没有“可操作性”、然而,它通过将启动“非和平方式”的条件放在“主观解释第8条”的基础上,更显出它是一个更具“攻击性”的法律。

    如此一来,华盛顿并没有因《反分裂国家法》的通过、就彻底地拿到北京的“底牌”。在东方评论员看来,
    \red{如果华盛顿能够拿到北京的“底牌”、那么,它就可以在不触踫这些“底牌”之前,大可以根据自己的全球战略需要、特别是中东进程的实际情况,精心地准备它的“台独时间表”。}
 

\item 《反分裂法》对“两岸现状”的“定义”、实际上就会被“一连串的法律和协议”“承认”下来\\
\label{sec-3_2_5}%
由于“反分裂国家法”将台海现状以“大陆和台湾同属一个中国”在法律上明确规范下来,并通过对“台独事实、台独重大事变、以及和平统一的可能性是否完全丧失”“三条件”也明确了自己的“界定方式”、从而首次将中国政府的旨在维护国家统一的“战争意志和决心”在法律上加以了“确定”。

    所以,不论是从立法的“时机与动机”、还是执法的“决心与意志”来看,东方评论员认为,《反分裂国家法》的正式文本的通过并立刻生效、显然是北京愿为台湾问题“付出一切代价”的最新例证。

    在东方评论员看来,对“两岸现状”给出清楚定义的第二条是非常重要的,可以说它对北京是重要的,对华盛顿也是重要的。我们认为,一旦华盛顿最终让步,并默认了北京所定义的“两岸现状”,那么,在首席评论员看来,有一点是非常清楚的,那就是,《反分裂法》对“两岸现状”的“定义”、实际上就会被 “一连串的法律和协议”“突然”地“承认”下来。
 

\item 华盛顿“玩转”“与台湾关系法”的“操作手法”、实际上已经“成为了历史”\\
\label{sec-3_2_6}%
我们认为,这“一连串的法律和协议”,不仅有已经生效的、给出“两岸现状”之“定义”的中国国内法--“反分裂国家法”,还有美国国内法-- “与台湾关系法”,最后,可以肯定的是,如果大陆和北京间最后能谈出个“中程协议”来的话,那么,这个“中程协议”将毫无例外地“继承”这一“定义”。

    如此一来,不论华盛顿是否愿意,事实上,它之前动不动就亮出“与台湾关系法”、并根据一时需要、或者刻意去“模糊”、或者有意去“明晰”那些个“介入台海的条件”之“操作手法”、实际上已经“成为了历史”。
 

\item 没有明列出“具体的红线”的负面影响也不容忽视\\
\label{sec-3_2_7}%
说到这里,我们再回过头去看“两岸现状”的“如何界定”的问题,我们应该可以更清楚地看到其重要意义之所在。显然,在东方评论员看来,“两岸现状”的“如何界定”的问题,直接关系到中国“能否合法启动”“反分裂国家法”,美国“能否顺利地”启动“与台湾关系法”的问题。

    就如我们昨天在有关台湾问题的点评中所说的那样,不难看出“修正后”的“正式文本”,不仅没有明列出“具体的红线”,较之“草案”相比,也更加体现出了一种特性,那就是“最大的弹性”。

    可以这样说,这种“将释法第八条”去“主观界定红线”的方式,由于没有明确定义所谓台湾不能跨越的“红线”,虽然留给大陆了更多灵活执法的空间,但其负面作用也不容忽视,那就是,极可能造成台湾方面的误判,并给了“台独”伺机挑畔《反分裂国家法》权威的空间。
 

\item 北京这种时刻准备着在“他日释法”“第八条”的态度,已经让方方面面感觉到了“台海的硝烟”\\
\label{sec-3_2_8}%
我们认为,不论是2000年发表的“白皮书”中所列明的“动武三前提”、还是前不久胡锦涛主席在“四点意见”中所列举的“台湾正名”、“去中国化”等“渐进式台独”形式、或者是人大副委员长王兆国在《反分裂国家法(草案)》的“说明”中、所明列的“几条”应引起高度警惕的“台独”手段,比如,台湾当局妄图利用所谓“宪法”和“法律”形式,通过“公民投票”、“宪政改造”等方式,其实“都有可能”促使人大根据《反分裂法》“第八条”解释出“台独事实”、“台独重大事变”的含义,人大甚至“还可能认为”、一旦出现这些情况之后,“和平统一的可能性也就完全丧失了”、从而、受《反分裂国家法》约束的中国政府、启动“非和平方式”也就是顺理成章的了。

    显然,这种“法律的强制力”是“看得见摸得着”的东西,也就是说,随着《反分裂国家法》的生效,尽管里面没有列明具体的红线,在东方评论员看来,北京似乎有意通过这种方式间接地为“他日释法”“第八条”“预做伏笔”。不难看出,北京这种时刻准备着在“他日释法”“第八条”的态度,已经让方方面面感觉到了 “台海的硝烟”。
 
 

\item 被迫“走回头路的”华盛顿、其“强硬的台湾政策”之“强硬度”,已经没有了足够的信用\\
\label{sec-3_2_9}%
因此,东方评论员认为,华盛顿在中东问题“前不着村后不着店”的无奈之下,为了避免“台独进程”与“反分裂法”迎头撞上,在台湾问题上只得逼迫“台独”“走走回头路”。

    如此一来,东方评论员认为,被迫“走回头路的”华盛顿、其“强硬的台湾政策”之“强硬度”,已经没有了足够的信用,在中国政府面前如此,在“台独”势力的面前也是如此,在日本、韩国、澳大利亚这些军事盟友面前又何尝不是如此呢?
 

\item 如何让人“相信”美国“也有一战的决心”,显然就成了华盛顿急需解决的问题\\
\label{sec-3_2_10}%
但是,值得强调的是,台湾无疑对美国具有重大的战略价值,然而,
    \red{这种战略价值之所以重大,很大程度上是因为遏制中国的战略需要。}
    对美国决策层而言,遏制中国只是其全球战略的一个重要方面,华盛顿眼中的威胁还有欧盟、俄罗斯、日本等等,这样,台湾的战略价值再怎么重要,也得着眼于其全球战略,换而言之,就算是中国统一了台湾,美国仍然可以是一个世界强权。

    但是,台湾对中国的意义就不一样了,彻底失去台湾的中国、由于将因之被迫付出极大的战略代价,实际上也就失去了成为世界强权的资格。因此,在台湾主权这个问题上,华盛顿很难向中国制定“反分裂国家法”那样、以一种可以为了台湾“不惜一切代价”的决心、去为其“与台湾关系法”树立起足够的威信。

    所以,在东方评论员看来,在一部刻意强调“台独”就是战争的“反分裂国家法”的面前,美国的“与台湾关系法”已经失去了往日的那种“随意解读”的空间。

    在这种尴尬之下,如何让人“相信”美国“也有一战的决心”,显然就成了华盛顿急需解决的问题。
 

\item 通过“打造”一个强大的“美日台”军事同盟的“实际行动”,去让人们重新相信美国“也有一战的决心”\\
\label{sec-3_2_11}%
有分析就认为,《反分裂国家法》的威信就来源于其立法的宗旨、而立法的宗旨就是以法律的强制力、约束中国政府必须以包括“非和平方式”在内一切手段、确保台湾留在中国。

    而华盛顿在“与台湾关系法”的权威、因自身在“是否介入台海”的问题上刻意“模糊”、被已经生效的、立场鲜明的《反分裂国家法》极大地冲击之后,华盛顿显然有意以打造一个强大的“美日台”军事同盟的“实际行动”,去让人们重新相信美国“也有一战的决心”。

    事实上,在华盛顿的对台政策的调整下,台湾、日本的军方都是积极配合,在东方评论员看来,陈水扁当局在这个时候搞“汉光二十一号演习”,并透露说“美日台”计算机兵棋联机也是已成事实,这只是“台独”和支持“台独”的国际势力、为避开“反分裂法”的锋芒、而准备“武力拒统”一系列动作中的一步而已。
 

\item 现实情况下,华盛顿眼中的“上佳的选择”\\
\label{sec-3_2_12}%
可以想像的是,一旦中国认为已经出现了启动“非和平方式”的条件,那么,台海战争已经成了事实。到时,华盛顿就是想根据“与台湾关系法”去介入台海,那也得考虑清楚:首先,由于中美都是“深具战争潜力”的大国,因此,双方如果直接交手、规模小了解决不了问题,但是规模大了,在中美两国都是核大国的情况下,进行一场大规模的战争、就有升级到核战争的风险,这对中美两国而言都是个灾难,这显然也不是个选择。

    因此,在东方军事评论员看来,在华盛顿决策层的眼中,眼下、在华盛顿准备继续向中东方向增加投放战略资源、而美军力量却出现“捉襟见肘”的情况下、全面提升台军的战力、让一个被美军高度整合,高度武装的“台独”去“武力拒统”、在眼前有助于帮美国维持台海现状,在将来又可大幅提高中国完成统一的战略成本,对华盛顿而言,这样去打算盘,不论是着眼于帮助自己去“遏制中国”、还是为了“消耗中国综合国力”、更或是赚取“现实的”商业利润、无疑都是 “上佳的选择”。

    \href{http://www.dongfangtime.com}{《东方时代环球时事解读》}





\end{itemize} % ends low level
\end{itemize} % ends low level
\section{2005年03月16日 星期三}
\label{sec-4}


  
\begin{itemize}

\item 朝称不能背着“暴政据点”恶名参加六方会谈\\
\label{sec-4_1}%
【平壤消息】朝鲜外务省发言人16日发表声明说,如果美国不收回对朝鲜“暴政据点”的侮辱性言论,朝鲜参加六方会谈将是“无法想象”的。

   发言人说,赖斯最近在接受英国路透社和美国《华盛顿时报》采访时表示,不会为自己称朝鲜为“暴政据点”而道歉。

   发言人说,美国一方面称自己的谈判伙伴为“暴政据点”而且拒不收回此种言论,一方面又希望自己的谈判伙伴尽快返回六方会谈,这本身就不合逻辑,实际上与放弃六方会谈也没什么两样。如果美国真想重开六方会谈,就必须收回这种侮辱性的言论,以现实、理智态度行事。

   发言人指出,赖斯的言词再一次暴露了美国不愿与朝鲜和平共处、试图孤立和扼杀朝鲜的对朝敌视政策。正因为如此,朝鲜为捍卫自己的政权,维护地区和平与稳定而继续加强自卫性核武库的行动是完全正当的。

   发言人还劝告美国,任何试图通过第三方来向朝鲜施压的想法都是愚蠢的,任何压力都不会使朝鲜改变自己的立场。


   【时事点评】在东方评论员看来,平壤所说的第三方当然是指的北京。我们知道,美国国务聊赖斯又要到北京进行访问了,我们还知道,这个“暴政据点”(暴政据点前哨)的“新名词”本身就是赖斯于1月18日、在正式上任前、于美国国会做听证时率先抛出来的。
 
\begin{itemize}

\item 朝鲜在这个时候提这个要求、其针对性是可想而知的\\
\label{sec-4_1_1}%
因此,在我们看来,朝鲜在这个时候提这个要求、要求美国收回对朝鲜“暴政据点”的侮辱性言论,并以此作为“回到六方会谈”的先决条件之一,其针对性是可想而知的。
 
   然而,在东方评论员看来,朝鲜是否参加六方会谈、其实早已经不是一个朝鲜自己能说了就算的事情了。这就是说,假如华盛顿“能成功地”通过“第三方”来对朝鲜施加强大的压力、那么,朝鲜“不参加”六方会谈也将是“无法想象”的、可问题是,华盛顿能够成功地“游说”北京、通过北京说服平壤回到“六方会谈”吗?

   在进行进一步展开之前,我们认为很有必要再回过头来看看
   \red{与朝鲜问题“脉动相联的”、台湾问题}
   上的一些最新进展。

   《东方时代环球时事解读》\href{http://www.dongfangtime.com}{http://www.dongfangtime.com}

\end{itemize} % ends low level

\item 华盛顿对《反分裂国家法》口头表示“不悦”\\
\label{sec-4_2}%
【华盛顿消息】据报道,美国国务院东亚助卿被提名人希尔今日表示,中国制定“反分裂国家法”是一种片面行动,美国促请两岸不要以任何片面行动改变现状。在参议院外交委员会的提名听证会上,希尔答询时说,“反分裂法”于事无补,美国坚持台海争议唯一解决方式是经由和平途径,美国反对使用武力。

   议员问到“反分裂法” 是否涉及改变现状,他回答说,在实质上,这是一种片面行动(approach),美国持续敦促任何一方不要尝试以片面行动改变现状。他重申美国的一贯政策,包括一个中国政策、三个公报、《台湾关系法》以及不支持台湾“独立”。

   国务院例行演示文稿会上,发言人包润石重申上周说过的,反分裂法与最近两岸关系和缓的趋势背道而驰,只会使立场更加强硬。他说,“我们反对任何以和平以外的方式试图决定台湾前途”。有人问及“美国是否会采取任何具体行动以表达不悦”时,包润石回答说,“我想我们刚刚做了”。意指刚才表达立场即是。

   有人问道台湾要在三月廿六日示威抗议一事。包润石表示,他对这些抗议、反应并无任何特别的看法。白宫发言人麦克雷兰则在例行演示文稿会上表示,这项法律是个“不幸”,对台海和平与稳定没有助益,“我们继续鼓励两岸对话,我们相信这对确保和平稳定、降低紧张,非常重要”。

   他说,片面改变现状会升高紧张,所以任何一方都不应如此。
 

   【时事点评】在东方评论员看来,美国官方的此番表态,事实上是继中国通过《反分裂国家法》,美国政府十四日做出所谓的“不幸”、“于事无补”这一类的“不疼平痒的关切”之后的继续。
 
  事实上,我们从来就“没有指望过华盛顿会喜欢”这部《反分裂法》,只是在东方评论员看来,华盛顿表示出“不悦”的本身、不过是为了向台湾岛内“准确地传送”一些北京所说的“错误信号”、而让台湾社会相信:美国“绝对不会”放弃“台湾”,因此,在这个问题上,我们今天不打算多说。在东方评论员看来,倒是美国人围绕着“反分裂国家法”、前前后后所选择的一系列表达“不悦”的“方式和措词”、倒值得我们评上一评。
 
\begin{itemize}

\item 美国“最反对”的就是北京强调“非和平方式”“阻独”的做法\\
\label{sec-4_2_1}%
就在全国人大常委会上周公布反分裂国家法“草案”,确认将以“非和平方式”处理台独问题后,我们注意到,美国政府的“统一口径”是:除重申“反对两岸任一方采取改变现状的片面行动外”,就是“无助两岸关系”、再就是“与当前两岸关系和缓的趋势背道而驰。”

    之后,白宫当时还将“非和平方式”形容成是对台湾的“惩罚性措施”,并以此“呼吁北京重新考虑”。而那位对台湾向来“非常同情”的美国务院亚太副助卿薛瑞福曾经在“反对”“反分裂法”的问题说了最尖锐的一个词,那就是公开批评反分裂法是“错误”,还要求大陆“设法修补”。

    在东方评论员看来,美国“最反对”的就是北京强调“非和平方式”“阻独”的做法,然而,就在以授权中央军委、国务院有“启动”“非和平方式”为手段、去阻止国家分裂为核心的《反分裂国家法》如期通过之后,就在台湾有人盼望着美国能有“更进一步、更深一层”表示“不悦”的时候,结果、我们看到的是,美国人最后吐出来的不过是“不幸”这个词。
 

\item 非常值得我们注意的一个细节\\
\label{sec-4_2_2}%
我们认为,尽管华盛顿在公开场合,从来都是以一种“反对”“反分裂法”的面目出现的,然而,直到今天为止,尽管美国政府的“反对”情绪是跃然纸上,但却从来没有将“反对”这个词正式“说出口”。在我们看来,这是非常值得我们注意的一个细节。
 

\item 不把“反对”这个词正式“吐出来”,华盛顿可以说是“用心老到”\\
\label{sec-4_2_3}%
首席评论员指出,华盛顿之所以处处表现出“反对”“反分裂法”、但却从不把“反对”这个词正式“吐出来”,可以说是“用心老到”,在我们看来,这表面在这几层“用心”。

    第一,之前,我们曾经说过,对北京这部旨在
    \red{干扰美国全球战略部署}
    的“反分裂国家法”,华盛顿之所以嘴里说不出什么正式“反对”的话来,主要原因在于有自知之明:即“就是反对也无济于事”。
 

\item “搞政治”的人、就得面对现实\\
\label{sec-4_2_4}%
第二,东方评论员认为,既然北京此次立法“反分裂国家法”是着眼于全球战略,那么,华盛顿的决策层应该明白这么个道理,那就是,既然挡不住“反分裂国家法”的“正式出炉”,那么,“搞政治”的人、就得面对现实,即,必须在北京用《反分裂国家法》打乱了自己
    \red{原先计划好了的“台海政策”}
    的基础上、重新去和北京在台海局势、东亚安全、乃到亚洲安全、中东问题上“有话好好说”。

    显然,如此一来,美国如果对一部“势必通过”、或者“已经通过”的《反分裂法》做“正式”、但却注定是“无效”的“反对”,那么,美国的“反对”就真成了 “错误的信号”,必将激起“台独”势力对《反分裂法》的极度反弹,也就等于将“中美之间”在一系列战略问题上“有话好好说”的基础、给“正式地、毫无效果地反对掉了”。不难看出,经验老到的华盛顿,这次没有给自己找这样一个麻烦。

    在东方评论员看来,上面所说的朝鲜问题,也正是华盛顿准备与北京在《反分裂国家法》已经“正式生效”的基础上,首先要谈到的一个“战略合作”课题。
 

\item 朝鲜问题事实上已经从一个“可控的热点”、被彻底“鼓捣成了”一个牵一发而动全身的“火山口”\\
\label{sec-4_2_5}%
然而,华盛顿眼中的“朝鲜核问题”、早已经被北京“另有目的”地“定义”成了“朝鲜半岛核问题”,而这其中、还被“夹塞进了”东亚核竞赛、美日安保条约的问题,总而言之,在我们看来,现在的朝鲜核问题,经被华盛顿“怂恿日本在钓鱼岛问题上挑衅中国”、拿到了一份包含台湾问题的“美日安全共同声明”之后,又经朝鲜突然“宣布有核”、北京顺势抓起了一张“东亚核竞赛”的牌、事实上已经从一个“可控的热点”、被彻底“鼓捣成了”一个牵一发而动全身的“火山口”。
 

\item 小泉政府这次是“唯恐东亚不乱”\\
\label{sec-4_2_6}%
东方评论员认为,由于日本在钓鱼岛、独岛(日本称竹岛)、北方四岛等一系列主权争端问题上、是全线出击、“挨着个地”寻畔几个核国家:先挑畔中国、再激怒韩国、朝鲜、最后还咬住了俄罗斯、在我们看来,小泉政府这次是大有“唯恐东亚不乱”、好借机
    \red{扩充军备、特别是核军备}
    、且“不达目的不收兵”的气慨。
 
 

\item 日本当局的主要着眼点,还是在于为日本今后的国际地位“寻畔出”一条捷径来\\
\label{sec-4_2_7}%
事实上,在日本国内,小泉纯一郎的日子并不好过,因此,借助“领土争端”这种手段,来加强自己的政治声望无疑是一大目的,然而,根据我们的观察,日本当局显然将目前这种美国陷于中东、有求于日本,中国顾忌亚洲经济一体化进程,顾忌“中日相争”而让美国渔翁得利、也不方便与日本彻底翻脸之际,当成了实现其“正常国家”的天赐良机。
 

\item 日本眼中的“捷径”,就是成为一个核大国\\
\label{sec-4_2_8}%
因此,日本当局的主要着眼点,还是在于为日本今后的国际地位、“正常国家”“寻畔出”一条捷径来,在东方评论员看来,日本眼中的这条“捷径” 就是成为一个核大国,以日本的人口规模、经济实力和常规军力,一旦跨入了核门槛,什么“国际地位”、“正常国家”、“军事大国”都成了手到擒来的东西,但是,这条捷径能否走得通,似乎在日本人看来,关键就在于能否让东亚局面“更加混乱”一些。

    有迹象显示,就在日本在钓鱼岛、独岛、北方四岛“拼命搞事”的时候,东亚的局势看起来的确比以前更加混乱了。下面,我们再来看一则“朝鲜外交部”的声明,之后,我们还会继续讨论这个话题。

    《东方时代环球时事解读》\href{http://www.dongfangtime.com}{http://www.dongfangtime.com}
   
\end{itemize} % ends low level

\item 朝鲜扬言已准备研制更多核武防卫美国的侵略\\
\label{sec-4_3}%
【平壤消息】据外电报道,朝鲜外交部表示,朝鲜拥有核武可以防止美国采取侵略行动,朝鲜已经准备研制更多核武。

   朝鲜官方通讯社引述外交部发言人指,事实证明,朝鲜拥有核武可以保证区内军力平衡,也可阻吓区内爆发战争,维护和平,朝鲜将会采取必须的反制措施,包括增加核武,对付极端仇视朝鲜,图谋推翻北韩现政权的美国。

   据路透社报道,朝鲜外交部发言人说:“事实表明,我们拥有的核武器能够确保地区力量平衡,是阻止战争爆发和维持和平的重要威慑力量。”发言人接着说:“朝鲜将会采取所有必要的应对措施,其中包括增强核武器力量,以应对美国终结朝鲜核计划的险恶用心。”


   【时事点评】显然,在东方评论员看来,东亚这块地方,要说乱也真够乱的。这边,是日本在那里“转着圈、挨着个”地得罪人,而且得罪的都是有核家伙的国家。
 
 
\begin{itemize}

\item 朝鲜似乎已经在为日本所渴望的“东亚核竞赛”放风\\
\label{sec-4_3_1}%
如果我们再仔细看看日本得罪的这些国家名字,就不难发现,不论是中国、韩国、朝鲜、俄罗斯、都是“六方会谈”中的“正式成员”,加上日本自己,还有那个美国,“无一不与核”沾边带故:要么本身就是身居“五大核国家”之列,如中、美、俄、要么就宣称自己“有了核武器”、比如朝鲜,最不济的也都有
    \red{立刻生产核武器能力}
    ,比如日本和韩国。

    因此,在这种情况下,我们再去品味朝鲜昨天所说的,“朝鲜拥有核武可以保证区内军力平衡,也可阻吓区内爆发战争,”的说法,就不难品出、朝鲜这似乎已经在为日本所渴望的“东亚核竞赛”放风了。
 

\item 突然抛出这番“重磅言论”,显然是在有意“砸”华盛顿\\
\label{sec-4_3_2}%
在我们看来,朝鲜外交部在局势如此敏感的时刻,突然抛出这番“重磅言论”,显然是在有意“砸”华盛顿、而且是有意“砸给”日本人看的:美国国务聊赖斯将到北京,其重点就是朝鲜问题,平壤在这个时候“放炮”,与上面所强调的、“美国任何试图通过第三方来向朝鲜施压的想法都是愚蠢的”的说法,其意图可以是一脉相承的。
 
 

\item 平壤这样做的最直接好处,可直接“拉高”华盛顿与北京“讨价还价”的“困难度”\\
\label{sec-4_3_3}%
在东方评论员看来,就如华盛顿近日来默许陈水扁当局、有限度地“反弹”“反分裂法”、有意给赖斯的北京之行制造筹码一样,北京默许朝鲜这样做的最直接好处,就是可以直接“拉高”华盛顿与北京“讨价还价”的“困难度”。

    要知道,华盛顿之所以在“美日安全共同声明”的问题上“首鼠两端”,赖斯在这个时候跑到北京来,不就是怕亚洲安全框架崩溃、不就是怕北京手中的那张“东亚核竞赛”牌打出去了吗?
 

\item 首次将之与“区内”军力平衡“挂钩”,是在“不怀好意”地“哪壶不开提哪壶”\\
\label{sec-4_3_4}%
现在,朝鲜再提“继续加强自卫性核武库”,并首次将之与“区内”军力平衡“挂钩”,分明是在冲着“美日之间”的那个、名为“同盟”、实为“主仆”的“军事同盟”关系、在“不怀好意”地“哪壶不开提哪壶”。
 

\item 在东亚的这局牌局中、真正玩牌的只是北京和华盛顿\\
\label{sec-4_3_5}%
东方评论员认为,就目前的情况看,似乎离日本当局所期望的局面更近了些,然而,一直在“处心积虑”地寻求“借机发挥”的日本人、也必须看到这一层,既,在东亚的这局牌局中、真正玩牌的就是北京和华盛顿。

    在我们看来,对日本而言,很不幸的是,玩牌的这两位尽管在一系列战略利益上“拳打脚踢”、争得厉害,但是,在如何控制、压制日本的问题上,只要有可能,双方都是绝对有意继续合作的。
 

\item 赖斯的此次来访,就是为了商讨出一个双方都能接受的办法\\
\label{sec-4_3_6}%
因此,日本要想指望朝鲜能说出、或者做出更能让日本“借机发挥”的事、或者话来的话,还得看看中美这一轮“战略对话”、或者“战略争吵”、能谈出、或者吵出个什么结果来。

   在我们看来,由于“东亚安全”早已经装着“两大热点”、一个是台湾、再一个就是朝鲜,因此,总体上看,由于北京和华盛顿“在实力上的碰撞”,北京一时解决不好华盛顿插手的台湾问题,而华盛顿面对一个北京公开支持的朝鲜,也是一时拿不出“可以如愿”解决朝鲜问题的高招。

    结果是,由于中美都有意让东亚局势服务于大局,服务于自己的全球战略,因此,在我们看来,在某种程度上,正因为北京和华盛顿在“朝鲜、台湾两个问题”上同时“对抗”、反而容易让中美两国“在对抗的气氛中”、寻找出一个“大体维持局面”、却能“各取所需的”方法来。
 

\item 如何敲打日本,或者干脆继续放纵日本,是北京必然会对华盛顿提出的一个选择题\\
\label{sec-4_3_7}%
东方评论员认为,赖斯的此次来访,就是为了商讨出一个双方都能接受的办法。然而,根据我们的观察,赖斯的此番重头戏之一的印度之行、并不理想,尽管印度和美国之间谈成了针对中国的“毫无内容”的“加强防务合作”的意向,但是,印度和美国之间、在伊朗问题上的矛盾却被进一步激化了。

   因此,为集中精力投身于中东事务、想安排好针对中国的、南起印度、东至日本的、意在防堵北京借机挤压美国亚洲利益的华盛顿,带着印度之行的“不愉快”,要想成功地“游说”北京、通过北京说服平壤回到“六方会谈”、仅仅揣着一个“不正式反对”“反分裂法”、或者“压制”陈水扁当局不得做出极端的“反制”行为、来权当礼物,恐怕是远远不够的。

   在东方评论员看来,如何敲打日本,或者干脆继续放纵日本,是北京必然会对华盛顿提出的一个选择题。在我们进一步展开讨论华盛顿可能会选择哪个选择项之前,我们再来看看两则消息。

   《东方时代环球时事解读》\href{http://www.dongfangtime.com}{http://www.dongfangtime.com}

   










\end{itemize} % ends low level
\end{itemize} % ends low level

\end{document}
